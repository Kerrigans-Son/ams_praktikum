\documentclass{IEEEtran}
\usepackage{lipsum}
\usepackage{cite,graphicx,amssymb,amsfonts,booktabs,multirow,array,comment}
\usepackage[cmex10]{amsmath}
\usepackage[caption=false,font=footnotesize]{subfig}
\usepackage[all,graph]{xy}
\title{Starcraft 2 Reinforcement Learning}
\author
{\IEEEauthorblockN{Ibrahim Cinar, Tobias Grasmeyer, Artur Schmidt} \\
\IEEEauthorblockA{Fachbereich Informatik\\
Hochschule Darmstadt}
}

\begin{document}
\maketitle
\begin{abstract}
\lipsum[1]
\end{abstract}
\section{Einführung}
\lipsum[1]
\lipsum[1]
\section{Titel2}
\lipsum[1]
\lipsum[1]
\lipsum[1]
\begin{figure}[!t]
\centering
\includegraphics[width=2.5in]{test.png}
\caption{Simulation Results}
\label{fig_sim}
\end{figure}
\lipsum[1]
\lipsum[1]
\lipsum[1]
\section{Conclusion}
\lipsum[1]
\lipsum[1]
\begin{table}
\renewcommand{\arraystretch}{1.3}
\caption{Simple table}
\label{tab:example}
\centering
\begin{tabular}{c|c}
    \hline
    Heading One  &  Heading Two\\
    \hline
    \hline

    Three   &   Four\\
    \hline

    Five    &   Six\\
    \hline
\end{tabular}
\end{table}
\lipsum[1]
\lipsum[1]
\begin{equation}
    \label{eq:kinetic_energy}
    E_{k} = \frac{1}{2}mv^{2}
\end{equation}
\lipsum[1]
\end{document}
